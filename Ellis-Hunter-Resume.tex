\documentclass[10pt, letterpaper]{article}

% Packages:
\usepackage[
    ignoreheadfoot, % set margins without considering header and footer
    top=1.0 cm, % seperation between body and page edge from the top
    bottom=1.0 cm, % seperation between body and page edge from the bottom
    left=1.5 cm, % seperation between body and page edge from the left
    right=1.5 cm, % seperation between body and page edge from the right
    footskip=1.0 cm, % seperation between body and footer
    % showframe % for debugging 
]{geometry} % for adjusting page geometry
\usepackage{titlesec} % for customizing section titles
\usepackage{tabularx} % for making tables with fixed width columns
\usepackage{array} % tabularx requires this
\usepackage[dvipsnames]{xcolor} % for coloring text
\definecolor{primaryColor}{RGB}{0, 0, 0} % define primary color
\usepackage{enumitem} % for customizing lists
\usepackage{amsmath} % for math
\usepackage[
    pdftitle={Hunter Ellis's Resume},
    pdfauthor={Hunter Ellis},
    pdfcreator={LaTeX with RenderCV},
    colorlinks=true,
    urlcolor=primaryColor
]{hyperref} % for links, metadata and bookmarks
\usepackage[pscoord]{eso-pic} % for floating text on the page
\usepackage{calc} % for calculating lengths
\usepackage{bookmark} % for bookmarks
\usepackage{lastpage} % for getting the total number of pages
\usepackage{changepage} % for one column entries (adjustwidth environment)
\usepackage{paracol} % for two and three column entries
\usepackage{ifthen} % for conditional statements
\usepackage{needspace} % for avoiding page brake right after the section title
\usepackage{iftex} % check if engine is pdflatex, xetex or luatex
\usepackage{datetime}

% Ensure that generate pdf is machine readable/ATS parsable:
\ifPDFTeX
    \input{glyphtounicode}
    \pdfgentounicode=1
    \usepackage[T1]{fontenc}
    \usepackage[utf8]{inputenc}
    \usepackage{lmodern}
\fi

\usepackage{charter}

% Some settings:
\raggedright
\AtBeginEnvironment{adjustwidth}{\partopsep0pt} % remove space before adjustwidth environment
\pagestyle{empty} % no header or footer
\setcounter{secnumdepth}{0} % no section numbering
\setlength{\parindent}{0pt} % no indentation
\setlength{\topskip}{0pt} % no top skip
\setlength{\columnsep}{0.15cm} % set column seperation
\pagenumbering{gobble} % no page numbering

\titleformat{\section}{\needspace{4\baselineskip}\bfseries\large}{}{0pt}{}[\vspace{1pt}\titlerule]

\titlespacing{\section}{
    % left space:
    -1pt
}{
    % top space:
    0.2 cm
}{
    % bottom space:
    0.2 cm
} % section title spacing

\renewcommand\labelitemi{$\vcenter{\hbox{\small$\bullet$}}$} % custom bullet points
\newenvironment{highlights}{
    \begin{itemize}[
        topsep=0 pt,
        parsep=0 pt, 
        partopsep=0pt,
        itemsep=0pt,
        leftmargin=0 cm + 10pt
    ]
}{
    \end{itemize}
} % new environment for highlights


\newenvironment{highlightsforbulletentries}{
    \begin{itemize}[
        topsep=0.10 cm,
        parsep=0.10 cm,
        partopsep=0pt,
        itemsep=0pt,
        leftmargin=10pt
    ]
}{
    \end{itemize}
} % new environment for highlights for bullet entries

\newenvironment{onecolentry}{
    \begin{adjustwidth}{
        0 cm + 0.00001 cm
    }{
        0 cm + 0.00001 cm
    }
}{
    \end{adjustwidth}
} % new environment for one column entries

\newenvironment{twocolentry}[2][]{
    \onecolentry
    \def\secondColumn{#2}
    \setcolumnwidth{\fill, 4.5 cm}
    \begin{paracol}{2}
}{
    \switchcolumn \raggedleft \secondColumn
    \end{paracol}
    \endonecolentry
} % new environment for two column entries

\newenvironment{threecolentry}[3][]{
    \onecolentry
    \def\thirdColumn{#3}
    \setcolumnwidth{, \fill, 4.5 cm}
    \begin{paracol}{3}
    {\raggedright #2} \switchcolumn
}{
    \switchcolumn \raggedleft \thirdColumn
    \end{paracol}
    \endonecolentry
} % new environment for three column entries

\newenvironment{header}{
    \setlength{\topsep}{0pt}\par\kern\topsep\linespread{1.5}
}{
    \par\kern\topsep
} % new environment for the header

% save the original href command in a new command:
\let\hrefWithoutArrow\href

\begin{document}
    \newcommand{\AND}{\unskip
        \cleaders\copy\ANDbox\hskip\wd\ANDbox
        \ignorespaces
    }
    \newsavebox\ANDbox
    \sbox\ANDbox{$|$}
    \noindent\huge{\textbf{Hunter Ellis}}\normalsize\hfill \hrefWithoutArrow{tel:+1-703-953-6963}{(703) 953-6963}\\ 
    \noindent\large{\textbf{Electrical \& Computer Engineer}}\normalsize\hfill\hrefWithoutArrow{mailto:hunterellis@vt.edu}{hunterellis@vt.edu}
    
    \kern 3.0 pt%
    \hrule
    \kern 5.0 pt%
    Engineer with an interest in control theory and embedded systems. \hfill \hrefWithoutArrow{https://github.com/hunterwellis}{{github.com/hunterwellis}}\\
    \quad$\hookrightarrow$ Currently, working on my Master's Thesis in \href{https://coolautonomylab.github.io/team/}{{Dr.Thinh Doan's Research Group}}.\hfill\hrefWithoutArrow{https://ellishw.tech}{{ellishw.tech}}\\

    \section{Skills}
    \textbf{Languages: }C++, Python, MATLAB, Embedded C, LaTeX, Verilog \\
    \kern 5.0 pt%
    \textbf{Tools: }Simulink, Soldering, PCB Design and Assembly, GNU/Linux, Git, ROS2, Gazebo, PyTorch, OpenCV, 3D-Printing,\\ 
    \quad\quad\quad SciKit-Learn, Make, CMake, LabView, Qt, KiCAD, FreeRTOS, Autodesk Inventor, SolidWorks, Rhino
    \section{Education}
    \begin{twocolentry}{{May 2025}\\\textit{Blacksburg, Virginia}}
        \textbf{Master of Science in Computer Engineering}\\
        Virginia Tech -- Focused on Control Theory\\ 
        \quad\quad\textit{Advisers: \hrefWithoutArrow{https://coolautonomylab.github.io/members/thinh.html}{{Dr.Thinh Doan (UT Austin)}} and \hrefWithoutArrow{https://filebox.ece.vt.edu/~mhsiao/}{{Dr.Michael Hsaio (Virginia Tech)}}}\\
    \end{twocolentry}
    \kern 5.0 pt%
    \begin{twocolentry}{{May 2024}\\\textit{Blacksburg, Virginia}}
        \textbf{Bachelor of Science in Electrical \& Computer Engineering} (double major)\\
        Virginia Tech -- Control Systems and Machine Learning
    \end{twocolentry}
    
    \section{Experience}
    \begin{twocolentry}{{Aug 2023 -- Present} \\\textit{Blacksburg, Virginia}}
        \textbf{Robotics \& Control Theory Research | M.S. Thesis}\\
        \textit{Virginia Tech \boldmath$\cdot$ Graduate Researcher}\\
        \begin{onecolentry}
            \begin{highlights}
                \item Undergraduate and graduate research developing neuro-symbolic algorithms
                \item Developing robotic hardware and software tools for testing algorithms
                \item Working on merging symbolic programming with deep RL for multi-task agents
                \item Building ROS2 workspaces and packages for training custom RL agents
                \item Designed and implemented RL methods to beat Atari games
            \end{highlights}
        \end{onecolentry}
    \end{twocolentry}
    \kern 5.0 pt%
    \begin{twocolentry}{{Aug 2024 -- Present} \\\textit{Blacksburg, Virginia}}
        \textbf{Graduate Teaching Assistant | Continuous \& Discrete Systems}\\
        \textit{Virginia Tech \boldmath$\cdot$ Teaching Assistant}\\
        \begin{onecolentry}
            \begin{highlights}
            \item Assisting Professors in teaching fundamental concepts in linear systems theory and DSP
            \item Holding office hours and preparing recitation sessions for students 
            \end{highlights}
        \end{onecolentry}
    \end{twocolentry}
    \kern 5.0 pt%
\begin{twocolentry}{{May 2024 -- Aug 2024} \\\textit{Huntsville, Alabama \\ (Merrit Island, Florida)}}
    \textbf{Thrust Vector Control | Mars Ascent Vehicle (MAV)}\\
        \textit{Jacobs Space Exploration Group \boldmath$\cdot$ TVC Intern}\\
        \begin{onecolentry}
            \begin{highlights}
            \item TVC for Mars Sample Return Mission and EUS at the NASA Marshall Space Flight Center
            \item Developed software and hardware systems for NASA’s Active Inertial Load Simulator
            \item Characterized dynamic systems for MAV’s TVC test stand using Python and MATLAB. 
            \item Derived a non-linear model and control architecture for a load simulating actuator 
            \item Traveled to Kennedy Space Center for the Space Launch System’s (Booster) TVC Testing
            \end{highlights}
        \end{onecolentry}
    \end{twocolentry}
    \kern 5.0 pt%
    \begin{twocolentry}{{June 2023 -- Aug 2023}\\ \textit{Grenoble, France}}
        \textbf{Control Systems Research | Microgrid Inverters}\\
        \textit{Grenoble Electrical Engineering Laboratory \boldmath$\cdot$ Research Intern}\\
        \begin{onecolentry}
            \begin{highlights}
            \item Researched "microgrids" -- designed to avoid infastructure problems on the French Grid
            \item Simulated neutral point balancing control methods using 4-leg inverters in Simulink
            \item Investigated NPC inverters with unbalanced network conditions for islanding events
            \end{highlights}
        \end{onecolentry}
    \end{twocolentry}
    \kern 5.0 pt%
    \begin{twocolentry}{{June 2022 -- Aug 2022}\\\textit{West Bethesda, Maryland}}
        \textbf{Naval Concept Design Research | Hospital Sea Trains}\\
        \textit{Naval Surface Warfare Center (Carderock Division) \boldmath$\cdot$ Concept Research Intern}\\
        \begin{onecolentry}
            \begin{highlights}
            \item Developed concept hospital sea-train design at the Center for Innovative Ship Design
            \item Estimated fuel consumption and electrical power loads of concept sea-trains 
            \end{highlights}
        \end{onecolentry}
    \end{twocolentry}
    \section{Projects}
    \begin{twocolentry}{{Dec 2023 -- Present}\\}
        \textbf{FOC Stepper Motor} (\hrefWithoutArrow{https://github.com/hunterwellis}{github.com/hunterwellis})
        \begin{onecolentry}
            \begin{highlights}
            \item Widely applicable stepper motor driver using FOC and a magnetic encoder for feedback
            \item 4-layer PCB mounts to the back of stepper with CAN and power connection
            \end{highlights}
        \end{onecolentry}
    \end{twocolentry}
    \kern 5.0 pt%
    \begin{twocolentry}{{Aug 2023 -- May 2024}\\}
    \textbf{Computer Vision | OCR Capstone Project }(\hrefWithoutArrow{https://ece.vt.edu/content/dam/ece_vt_edu/S24-ECE-MDE-Brochure.pdf}{capstone\_brochure.pdf})
        \begin{onecolentry}
            \begin{highlights}
            \item IOS application capable of detecting coins of interest/value
            \item Trained OCR and ResNet-50 models on dataset of real and augmented coin images
            \end{highlights}
        \end{onecolentry}
    \end{twocolentry}
    \kern 5.0 pt%
    \begin{twocolentry}{{Oct 2020 -- Mar 2023}\\}
        \textbf{Design Teams | Solar Car \& Human Powered Submarine }(\hrefWithoutArrow{https://solarcaratvt.org}{solarcaratvt.org})
        \begin{onecolentry}
            \begin{highlights}
            \item Overall E/E architecture of the Solar Car
            \item Single board computer and LCD to display relevant data to the submarine pilot 
            \end{highlights}
        \end{onecolentry}
    \end{twocolentry}
\end{document}
