\documentclass[9pt, letterpaper]{extarticle}

% Packages:
\RequirePackage{fontawesome}
\usepackage[
    ignoreheadfoot, % set margins without considering header and footer
    top=1.0 cm, % seperation between body and page edge from the top
    bottom=1.0 cm, % seperation between body and page edge from the bottom
    left=1.2 cm, % seperation between body and page edge from the left
    right=1.2 cm, % seperation between body and page edge from the right
    footskip=1.0 cm, % seperation between body and footer
    % showframe % for debugging 
]{geometry} % for adjusting page geometry
\usepackage{titlesec} % for customizing section titles
\usepackage{tabularx} % for making tables with fixed width columns
\usepackage{array} % tabularx requires this
\usepackage[dvipsnames]{xcolor} % for coloring text
\definecolor{primaryColor}{RGB}{0, 0, 0} % define primary color
\usepackage{enumitem} % for customizing lists
\usepackage{amsmath} % for math
\usepackage[
    pdftitle={Hunter Ellis's Resume},
    pdfauthor={Hunter Ellis},
    pdfcreator={LaTeX with RenderCV},
    colorlinks=true,
    urlcolor=primaryColor
]{hyperref} % for links, metadata and bookmarks
\usepackage[pscoord]{eso-pic} % for floating text on the page
\usepackage{calc} % for calculating lengths
\usepackage{bookmark} % for bookmarks
\usepackage{lastpage} % for getting the total number of pages
\usepackage{changepage} % for one column entries (adjustwidth environment)
\usepackage{paracol} % for two and three column entries
\usepackage{ifthen} % for conditional statements
\usepackage{needspace} % for avoiding page brake right after the section title
\usepackage{iftex} % check if engine is pdflatex, xetex or luatex
\usepackage{datetime}

% Ensure that generate pdf is machine readable/ATS parsable:
\ifPDFTeX
    \input{glyphtounicode}
    \pdfgentounicode=1
    \usepackage[T1]{fontenc}
    \usepackage[utf8]{inputenc}
    \usepackage{lmodern}
\fi

\usepackage{charter}

% Some settings:
\raggedright
\AtBeginEnvironment{adjustwidth}{\partopsep0pt} % remove space before adjustwidth environment
\pagestyle{empty} % no header or footer
\setcounter{secnumdepth}{0} % no section numbering
\setlength{\parindent}{0pt} % no indentation
\setlength{\topskip}{0pt} % no top skip
\setlength{\columnsep}{0.15cm} % set column seperation
\pagenumbering{gobble} % no page numbering

\titleformat{\section}{\needspace{4\baselineskip}\bfseries\Large}{}{0pt}{}[\vspace{1pt}\titlerule]

\titlespacing{\section}{
    % left space:
    -1pt
}{
    % top space:
    0.15 cm
}{
    % bottom space:
    0.15 cm
} % section title spacing

\renewcommand\labelitemi{$\vcenter{\hbox{\small$\bullet$}}$} % custom bullet points
\newenvironment{highlights}{
    \begin{itemize}[
        topsep=0 pt,
        parsep=0 pt, 
        partopsep=0pt,
        itemsep=0pt,
        leftmargin=0.25 cm + 10pt
    ]
}{
    \end{itemize}
} % new environment for highlights


\newenvironment{highlightsforbulletentries}{
    \begin{itemize}[
        topsep=0.10 cm,
        parsep=0.10 cm,
        partopsep=0pt,
        itemsep=0pt,
        leftmargin=10pt
    ]
}{
    \end{itemize}
} % new environment for highlights for bullet entries

\newenvironment{onecolentry}{
    \begin{adjustwidth}{
        0 cm + 0.00001 cm
    }{
        0 cm + 0.00001 cm
    }
}{
    \end{adjustwidth}
} % new environment for one column entries

\newenvironment{twocolentry}[2][]{
    \onecolentry
    \def\secondColumn{#2}
    \setcolumnwidth{\fill, 4.5 cm}
    \begin{paracol}{2}
}{
    \switchcolumn \raggedleft \secondColumn
    \end{paracol}
    \endonecolentry
} % new environment for two column entries

\newenvironment{threecolentry}[3][]{
    \onecolentry
    \def\thirdColumn{#3}
    \setcolumnwidth{, \fill, \fill, \fill}
    \begin{paracol}{3}
    {\raggedright #2} \switchcolumn
    \begin{center}  % center the middle column content
}{
    \end{center}
    \switchcolumn \raggedleft \thirdColumn
    \end{paracol}
    \endonecolentry
} % new environment for three column entries

\newenvironment{header}{
    \setlength{\topsep}{0pt}\par\kern\topsep\linespread{1.5}
}{
    \par\kern\topsep
} % new environment for the header

% save the original href command in a new command:
\let\hrefWithoutArrow\href

\begin{document}
    \newcommand{\AND}{\unskip
        \cleaders\copy\ANDbox\hskip\wd\ANDbox
        \ignorespaces
    }
    \newsavebox\ANDbox
    \sbox\ANDbox{$|$}
    \begin{threecolentry}
        {
            Blacksburg, Virginia\\
            \hrefWithoutArrow{tel:+1-703-953-6963}{(703) 953-6963}\\
            \hrefWithoutArrow{mailto:hunterellis@vt.edu}{hunterellis@vt.edu}
        }
        {
            \hrefWithoutArrow{https://ellishw.tech}{{ellishw.tech}}\\
            \hrefWithoutArrow{https://github.com/hunterwellis}{{github.com/hunterwellis}}\\
            \hrefWithoutArrow{https://www.linkedin.com/in/ellishw/}{linkedin.com/in/ellishw}
        }
        {
            \Huge\textbf{{Hunter Ellis}}\\
            \kern 3.0 pt%
            \LARGE{\textbf{Electrical \& Computer Engineer}}
        }
    \end{threecolentry}
    \hrule
    \kern 3.0 pt%
    Electrical/Computer Engineer with interests in control systems and signal processing. Currently working on my master's thesis in an accelerated program at Virginia Tech.
    \section{Education}
    \begin{twocolentry}{{May 2025}\\\textit{Blacksburg, Virginia}}
        \textbf{Master of Science in Computer Engineering}\\
        Virginia Tech -- Focused on Control Theory -- GPA: 3.7/4.0\\ 
    \quad\quad\textit{Advisers: }\hrefWithoutArrow{https://coolautonomylab.github.io/members/thinh.html}{{\textit{Dr.Thinh Doan (UT Austin)}}} and \hrefWithoutArrow{https://filebox.ece.vt.edu/~mhsiao/}{{\textit{Dr.Michael Hsiao (Virginia Tech)}}}\\
    \end{twocolentry}
    \kern 3.0 pt%
    \begin{twocolentry}{{May 2024}\\\textit{Blacksburg, Virginia}}
        \textbf{Bachelor of Science in Electrical \& Computer Engineering} (double major)\\
        Virginia Tech -- Control Systems and Machine Learning -- GPA: 3.7/4.0
    \end{twocolentry}
    
    \section{Technical Experience}
        \textbf{Virginia Tech \boldmath$\cdot$ Control Theory (Reinforcement Learning) Research}\\
    \begin{twocolentry}{{Aug 2023 -- Present} \\\textit{Blacksburg, Virginia}}
        \textbf{\textit{Graduate Researcher}}\\
        \begin{onecolentry}
            \begin{highlights}
                \item Undergraduate and graduate research developing neuro-symbolic reinforcement learning algorithms with \hrefWithoutArrow{https://coolautonomylab.github.io/}{The C.O.O.L. Autonomy Lab at UT Austin.}
                \item Developing hardware for a 6-axis robot arm and software for a ROS2 + Gazebo simulation environment used to test custom reinforcement learning algorithms.
            \end{highlights}
        \end{onecolentry}
    \end{twocolentry}
    \begin{twocolentry}{{Aug 2024 -- Present} \\\textit{Blacksburg, Virginia}}
        \textbf{\textit{Graduate Teaching Assistant}}\\
        \begin{onecolentry}
            \begin{highlights}
            \item Taught fundamental concepts in linear systems theory and digital signal processing, including Laplace Transforms, Z-Transforms, system stability, and FIR \& IIR filter design.
            \item Assisted with hands-on projects to illustrate and integrate analog and digital filter design and application on breadboards and TI MSP432 development boards.
            \end{highlights}
        \end{onecolentry}
    \end{twocolentry}
    \kern 3.0 pt%
    \textbf{Jacobs Space Exploration Group \boldmath$\cdot$ Mars Ascent Vehicle (MAV)}\\
\begin{twocolentry}{{May 2024 -- Aug 2024} \\\textit{Huntsville, Alabama \\ (Merrit Island, Florida)}}
    \textbf{\textit{Thrust Vector Control Intern}}\\
        \begin{onecolentry}
            \begin{highlights}
            \item Developed thrust vector control testing hardware and software for NASA's Active Inertial Load Simulator at the Marshall Space Flight Center.
            \item Characterized and created a model of an electro-mechanical actuator including internal viscous and (non-linear) coulomb friction components.
            \item Derived control systems for a load simulating actuator, in Simulink -- used to simulate external loads placed on the Mars Ascent Vehicle's thrust vector control actuators during flight. 
            \item Designed and integrated a 3\textsuperscript{rd} order IIR filter to remove high frequency noise from a load cell and linear variable differential transformer (LVDT).
            \end{highlights}
        \end{onecolentry}
    \end{twocolentry}
    \kern 3.0 pt%
        \textbf{Grenoble Electrical Engineering Laboratory \boldmath$\cdot$ Microgrid Inverters}\\
    \begin{twocolentry}{{Jun 2023 -- Aug 2023}\\ \textit{Grenoble, France}}
        \textbf{\textit{Control Systems Research Intern }}\\
        \begin{onecolentry}
            \begin{highlights}
            \item Researched inverter control systems -- designed to be robust to islanding events and avoid future infrastructure problems on the French power grid.
            \item Simulated neutral point capacitive and balancing control methods using 4-leg inverters in Simulink. Tested PI control, PR control, Clarke and Park Transforms with HIL simulations.
            \end{highlights}
        \end{onecolentry}
    \end{twocolentry}
    \kern 3.0 pt%
        \textbf{Naval Surface Warfare Center (Carderock Division) \boldmath$\cdot$ Hospital Sea Trains}\\
    \begin{twocolentry}{{Jun 2022 -- Aug 2022}\\\textit{West Bethesda, Maryland}}
        \textbf{\textit{Concept Research Intern}}\\
        \begin{onecolentry}
            \begin{highlights}
            \item Developed concept hospital sea-train designs at the Center for Innovation in Ship Design and estimated fuel consumption and electrical power loads of concept sea-trains.
            \end{highlights}
        \end{onecolentry}
    \end{twocolentry}
    \section{Skills}
    \textbf{Software: }C/C++, Python, MATLAB, GNU/Linux, Simulink, Git, ROS2, Gazebo, Make, CMake, Labview, Qt,\\
    \quad\quad\quad\quad\ \ \ PyTorch, OpenCV, LaTeX, Verilog, FreeRTOS, Autodesk Inventor (Certified), SolidWorks, Rhino\\
    \kern 3.0 pt%
    \textbf{Hardware: }PCB Design and Assembly, Breadboarding, Computer Architecture, Oscilloscope, Multimeter, 3D-Printing\\ 
    \section{Projects}
    \begin{twocolentry}{{Aug 2024 -- Present}\\}
        \hrefWithoutArrow{https://github.com/hunterwellis/Manipulator-Environment}{\textbf{6-Axis Robotic Arm} \faExternalLink}
        \begin{onecolentry}
            \begin{highlights}
            \item 3D printed robot arm, built using stepper motors and pulleys.
            \item ROS2 Jazzy control and Gazebo Harmonic simulation.
            \end{highlights}
        \end{onecolentry}
    \end{twocolentry}
    \kern 3.0 pt%
    \begin{twocolentry}{{Dec 2023 -- Present}\\}
        \hrefWithoutArrow{https://ellishw.tech/project-pages/FOC.html}{\textbf{Closed Loop Stepper Motor} \faExternalLink}
        \begin{onecolentry}
            \begin{highlights}
            \item Backdrivable stepper motor driver using closed loop control and a magnetic encoder for feedback.
            \item 4-layer PCB mounts to the back of the motor with CAN and power connections.
            \end{highlights}
        \end{onecolentry}
    \end{twocolentry}
    \kern 3.0 pt%
    \begin{twocolentry}{{Oct 2020 -- Mar 2023}\\}
        \hrefWithoutArrow{https://solarcaratvt.org}{\textbf{Design Teams | Solar Car \& Human Powered Submarine} \faExternalLink}
        \begin{onecolentry}
            \begin{highlights}
            \item Overall E/E architecture of the Solar Car.
            \item Single board computer and LCD to display relevant data to the submarine pilot.
            \end{highlights}
        \end{onecolentry}
    \end{twocolentry}
    \kern 3.0 pt%
    \begin{twocolentry}{{Aug 2023 -- May 2024}\\}
        \hrefWithoutArrow{https://ellishw.tech/assets/pdf/coin_detection.pdf}{\textbf{Optical Charcter Recognition Capstone} \faExternalLink}
        \begin{onecolentry}
            \begin{highlights}
            \item IOS application capable of detecting coins of interest/value.
            \item Trained OCR and ResNet-50 models on a dataset of real and augmented coin images.
            \end{highlights}
        \end{onecolentry}
    \end{twocolentry}
\end{document}
